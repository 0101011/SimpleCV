\documentclass[a4paper,landscape,8pt]{article}
\usepackage[landscape]{geometry}
\usepackage[utf8]{inputenc}

%opening
\title{Switching over to SimpleCV.}
\author{Copyright \copyright 2012 SimpleCV.\\}

\begin{document}

\maketitle

SimpleCV\footnote{References : O'Reilly Publication, Practical Computer Vision with SimpleCV by Nathan Oostendorp, Anthony Oliver, and Katherine Scott.}, which stands for Simple Computer Vision, is an easy-to-use Python frame-work that bundles together open source computer vision libraries and algorithms for solving problems. The idea of this document is to provide a quick reference for switching from Matlab and OpenCV to SimpleCV. 
\begin{flushleft}
\begin{tabular}{llll}
  \hline
  Description & Matlab & OpenCV & SimpleCV \\ \hline \\[.1cm] 
  Reading an image & imread(`lenna.png')  & cvLoadImage('lenna.png') & Image('lenna.png')\\ [0.3cm]
  Converting the image \\ to RGB colorspace & hsv2rgb(hsv\_image) or ind2rgb(X, map) & CvtColor(bitmap, retVal, CV\_BGR2RGB) & img.toRGB() \\ [0.4cm]
  Converting the image \\ to BGR colorspace & - & CvtColor(bitmap, retVal, CV\_RGB2BGR) & img.toBGR()\\ [0.4cm] 
  Converting the image \\ to HLS colorspace & - & CvtColor(bitmap, retVal, CV\_RGB2HLS) & img.toHLS()\\ [0.4cm]
  Converting the image \\ to HSV colorspace & rgb2hsv(rgb\_image) & CvtColor(bitmap, retVal, CV\_RGB2HSV) & img.toHSV()\\ [0.4cm]
  Converting the image \\ to XYZ colorspace & cform = makecform(`srgb2xyz'); applycform(rgb,cform);  & CvtColor(bitmap, retVal, CV\_RGB2XYZ) & img.toXYZ()\\ [0.4cm] 
  \end{tabular}
\begin{tabular}{llll}
  \hline
  Description & Matlab & OpenCV & SimpleCV \\ \hline \\[.1cm] 
  
  Converting the image \\ to GRAY colorspace & rgb2gray(rgb\_image) & CvtColor(bitmap, retVal, CV\_RGB2GRAY) & img.toGray()\\ [0.6cm]
  
  Create a new, empty \\ OpenCV bitmap & zeros(H, W, C) & SetZero(bitmap) & img.getEmpty(channels)\\ [0.8cm]
  
  Full copy of the image & newimg = img & Copy(bitmap, newimg) & img.copy()\\ [.4cm]
  
  Resize the image & imresize(img, scale) & Resize(bitmap, scaled\_bitmap) & img.resize(x,y) \\ [.4cm]
  
  Smooth the image & H = fspecial(\textit{type}); imfilter(I,H) & Smooth(r, ro, algorithm, win\_x, win\_y, &img.smooth(algorithm\_name, aperature,\\
  & & sigma, spatial\_sigma) &sigma, spatial\_sigma, grayscale) \\[.4cm]
  
  Invert image & imcomplement(img) & & img.invert() \\[.3cm]
  
  Horizontally mirror \\ an image & flipdim(img,2) &Flip(bitmap, newimg\_bitmap, 1) & img.flipHorizontal() \\[.3cm]
  
  Vertically mirror \\ an image & flipdim(img,1) &Flip(bitmap, newimg\_bitmap, 0) & img.flipVertical() \\[.3cm] 
  
  Stretch filter on  & img(img$<$th\_l) = 0; & Threshold(grayscale\_bitmap, \\
  a greyscale image&  img(img$>$th\_h) = 255 & newimg, thresh\_low, 255,CV\_THRESH\_TOZERO) &img.stretch(thresh\_low, thresh\_high) \\[0.3cm]
  
  Binary threshold & & Threshold(bitmap, bitmap,\\
  of the image& step(vision.Autothresholder,img) & thresh, maxv, CV\_THRESH\_BINARY\_INV) &img.binarize(thresh, maxv, blocksize, p)\\[.3cm]
  
  Mean color of the image& mean(reshape(im, size(im,1)*size(im,2), & &\\
  & size(im,3))) & cv.Avg(bitmap)[0:3] & img.meanColor()\\[0.3cm]
  
  Finds the FeatureSet  & & GoodFeaturesToTrack(GrayscaleBitmap, eig\_image, &img.findCorners(maxnum, minquality,\\ 
  strongest corners first &  corner(img) & temp\_image, maxnum, minquality, mindistance, None) &mindistance)\\[.4cm]
  
  Blobs are continuous & & &img.findBlobs(threshval, minsize, \\
  light regions&  step(vision.BlobAnalysis, fg\_img)  & &maxsize, threshblocksize, threshconstant) \\[0.3cm]
  
  Finding the location& & HaarDetectObjects(EqualizedGrayscaleBitmap(),  &findHaarFeatures(self, cascade, scale\_factor,\\
  of a known object& - & cascade.getCascade(), storage, scale\_factor, use\_canny) &  min\_neighbors, use\_canny) \\[0.3cm]
 
  Uploading the Image & & & \\
  to Imgur or Flickr & - & & img.upload(dest,api\_key,api\_secret,verbose) \\[0.3cm]   

 \hline
\end{tabular}
\begin{tabular}{llll}
  \hline
  Description & Matlab & OpenCV & SimpleCV \\ \hline \\[.1cm] 
  Draw a circle \\ on the Image & step(vision.MarkerInserter, img, pts) & & img.drawCircle(ctr, rad, color, thickness)\\ [0.3cm]
  Draw a line & plot(X\_vector, Y\_vector) & & img.drawLine(pt1, pt2, color, thickness) \\[0.3cm]
  Size of image & [size(img,1) size(img,2)] & GetSize(bitmap)&img.size()\\[0.3cm]
  Split the image into a & - & & img.split(cols, rows)\\
  series of image chunks \\[0.3cm]
  Split the channels & r=img(:,:,1); g=img(:,:,2); b=img(:,:,3) & Split(bitmap, b, g, r, None) &img.splitChannels(grayscale) \\
  of an image into RGB \\[0.3cm]
  Images of R,G,B channels \\
  are recombined into a & cat(3, r, g, b) &Merge(b,g,r,None,retVal) & img.mergeChannels(r,b,g) \\
  single image\\[0.3cm]
  Apply a color correction\\
  curve in HSL space& - & &img.applyHLSCurve(hCurve, lCurve, sCurve)\\[0.3cm]
  Apply a color correction\\ 
  curve in RGB space& - & &img.applyRGBCurve(rCurve, gCurve, bCurve)\\[0.3cm]
  Applies Intensity to \\
  all three color channels & - & & img.applyIntensityCurve(curve)\\[0.3cm]
  Returns image representing \\
  the distance of each pixel & - & &img.colorDistance(color) \\ 
  from a given color tuple \\ [0.3cm]
  Apply morphological\\ 
  erosion to a image& imerode(img,SE) & Erode(bitmap, retVal, kern, iterations) &img.erode(iterations) \\[0.3cm]
  Apply morphological\\ 
  dilation to a image& imdilate(img,SE) &Dilate(bitmap, retVal, kern, iterations) & img.dilate(iterations) \\[0.3cm]
\end{tabular}

\begin{tabular}{llll}
  \hline
  Description & Matlab & OpenCV & SimpleCV \\ \hline \\[.1cm] 
  Histogram equalization & histeq(img, hgram) & cv.EqualizeHist(GrayscaleBitmap, & img.equalize() \\
  on the image& &  Equalizedgraybitmap )  \\[0.3cm]
  Returns Image of\\ the string& - &  &img.toString() \\[0.3cm]
  Applies erosion operation &  & MorphologyEx(bitmap, retVal, & img.morphOpen()\\
  followed by a & imerode(img, SE) &  temp, kern, CV\_MOP\_OPEN, 1)  \\
  morphological dilation \\ [0.3cm]
  The difference between the\\
  morphological dilation and& & MorphologyEx(Bitmap, retVal, & img.morphGradient() \\
  the morphological gradient & - & temp, kern, CV\_MOP\_GRADIENT, 1) \\[0.3cm]
  1D histogram(numpy array) \\ 
  of intensity for pixels & step(vision.Histogram,img) & &img.histogram(numbins)\\
  in the image \\[0.3cm]
  The histogram of the hue \\
  channel for the image& - & &img.hueHistogram(bins)\\[0.3cm]
  Returns the peak hue \\
  values histogram of hues& - & &img.huePeaks(bins)\\[0.3cm]
  Add two images& imadd(img1,img2) &Add(imgBitmap, otherBitmap, newBitmap) &img.\_\_add\_\_(other) \\[0.3cm]
  Subtract two images& imsubtract(img1,img2) &Sub(imgBitmap, otherBitmap, newBitmap) &img.\_\_sub\_\_(other)\\[0.3cm]
  Or two images& - &Or(imgBitmap, otherBitmap, newBitmap) &img.\_\_or\_\_(other)\\[0.3cm]
  Image division operation\\
  taking two images as input & imdivide(img1,img2) &Div(imgBitmap, otherBitmap, newBitmap) &img.\_\_div\_\_(other)\\[0.3cm]
  
  Raises every array element\\
  in image array to a power& img.\^{}p & Pow(imgBitmap, otherBitmap, other) & img.\_\_pow\_\_(other)\\[0.3cm]
  
  \end{tabular}

\begin{tabular}{llll}
  \hline
  Description & Matlab & OpenCV & SimpleCV \\ \hline \\[.1cm] 
  
 Finds 2d and 1d \\
 barcodes in the image & - & & img.findBarcode(zxing\_path)\\[0.3cm]

 Finds line segments  & hough(BW) & HoughLines2(em, CreateMemStorage(),  &img.findLines(threshold, minlinelength, \\
 in the image & & CV\_HOUGH\_PROBABILISTIC, 1.0, CV\_PI/180.0, &maxlinegap, cannyth1, cannyth2)\\
 & & threshold, minlinelength, maxlinegap)\\[0.3cm]
 
 Finds a chessboard & - & FindChessboardCorners(EqualizedGrayscaleBitmap, & img.findChessboard(dimensions, subpixel)\\
 within that image & & dimensions, CV\_CALIB\_CB\_ADAPTIVE\_THRESH \\
 & & + CV\_CALIB\_CB\_NORMALIZE\_IMAGE )\\[0.3cm]
 
 Canny edge detection method& edge(img,`canny') &Canny(GrayscaleBitmap, edgeMap, t1, t2) & img.edges(t1, t2)\\[0.3cm]
 
 function rotates an image \\
 around a specific point & imrotate(img,angle) & GetRotationMatrix2D(point , angle, & img.rotate(angle, fixed, point, scale)\\ 
 by the given angle & & scale, rotMat)\\ [0.3cm]
 
 return a shear-ed image & tform = maketform(`affine',A);  & &\\ 
 from the cornerpoints& imtransform(img,tform) & GetAffineTransform(src, cornerpoints, aWarp) &img.shear(cornerpoints) \\[0.3cm]
 
 Function for warp & tform = maketform(`projective',A); & & \\
 performs an affine rotation & imtransform(img,tform) & cv.WarpPerspective(imgBitmap, retVal, rotMatrix) & img.transformPerspective(rotMatrix) \\[0.3cm]
  
 Returns the RGB value for\\
 a particular image pixel & img(y, x, :) & Get2D(Bitmap, y, x) &img.getPixel(x, y)\\[0.3cm] 
 
 Returns the gray value for \\ 
 a particular image pixel & gray=rgb2gray(img); gray(y,x) & Get2D(GrayscaleBitmap(), y, x) & img.getGrayPixel( x, y) \\[0.3cm]
 Returns a single column of \\
 RGB values from the image& squeeze(img(:, column, :)) & GetCol(imgBitmap, column) & img.getVertScanline(column) \\[0.3cm]
 Returns a single row of \\
 RGB values from the image& squeeze(img(row, :, :)) & GetRow(imgBitmap, row) & img.getHorzScanline(row)\\[0.3cm]
 \end{tabular}

\begin{tabular}{llll}
  \hline
  Description & Matlab & OpenCV & SimpleCV \\ \hline \\[.1cm] 

 Returns a single column of & gray=rgb2gray(img);  & & \\
 gray values from the image& squeeze(gray(:, column, :)) & GetCol(imgGrayscaleBitmap, column) & getVertScanlineGray(column)\\[0.3cm]
 
 Returns a single row of & gray=rgb2gray(img); & &\\
 gray values from the image& squeeze(gray(row, :, :)) & GetRow(imgGrayscaleBitmap, row) & getHorzScanlineGray(row)\\[0.3cm]
 
 Crops the image \\ 
 based on parameters& imcrop(img, rect) & & img.crop(x , y, w, h, centered)\\[0.3cm]
 
 Returns the selected region.& imrect(hparent,position) &&img.regionSelect(x1, y1, x2, y2 )\\[0.3cm]
 
 Clears out the entire image & img(:)=0 & SetZero(Bitmap) & img.clear()\\[0.3cm]
 
 Draws the string on the image \\
 at the specified coordinates.& text(x,y,`string')  &&img.drawText(text , x , y , color, fontsize)\\[0.3cm]
 
 Draw a rectangle on the image & rectangle(`Position',[x,y,w,h]) & & img.drawRectangle(x,y,w,h,color,width,alpha)\\[0.3cm]
 
 Shows the current image& imshow(img) & ShowImage("Image", image)& img.show(type)\\[0.3cm]
 
 Push a new drawing layer onto\\
 the back of the layer stack& - & & img.addDrawingLayer(layer)\\[0.3cm]
 Insert a new layer into the \\
 layer stack at the specified index & - & & img.insertDrawingLayer(layer, index)\\[0.3cm]
 Remove a layer from the layer \\
 stack based on the layer's index& - & &img.removeDrawingLayer(index)\\[0.3cm]
 Return a drawing layer\\ 
 based on the index& - & & img.getDrawingLayer(index)\\[0.3cm]
 Remove all of the drawing layers& - & &img.clearLayers()\\[0.3cm]
 Return the array of \\
 DrawingLayer objects & - & & img.layers()\\[0.3cm]
 Return all DrawingLayer \\
 objects as a single & - &&img.mergedLayers()\\
 DrawingLayer.\\[0.3cm]
 Render all of the layers \\
 onto the current image& - & & img.applyLayers(indicies)\\[0.3cm]
 \end{tabular}
\begin{tabular}{llll}
  \hline
  Description & Matlab & OpenCV & SimpleCV \\ \hline \\[.1cm] 
  automatically adjust image size \\
  to match the display size& & &img.adaptiveScale(resolution,fit=True)\\[0.3cm]

  Combine two images as \\
  a side by side images  & & & img1.sideBySide(img2, side, scale)\\[0.3cm]
  Generate a binary mask \\
  of the image based on a & & & createBinaryMask(self,color1,color2)\\
  range of rgb values \\[0.3cm]
  Make the canvas larger but \\
  keep the image the same size & & &img.embiggen(size, color, pos) \\[0.3cm]
  The white areas of the mask \\
  will be kept and the black  & & & img.applyBinaryMask(mask,bg\_color)\\
  areas removed\\[0.3cm] 
  Generate a grayscale or binary \\
  mask image based either on a & & & img.createAlphaMask(hue, hue\_lb,hue\_ub) \\
  hue or an RGB triplet\\[0.3cm]
  Apply a function to every pixel\\ 
  and return the result & & & img.applyPixelFunction(theFunc)\\[0.3cm] 
  Calculate the integral image\\ 
  and return it as a numpy array& & Integral(GrayscaleBitmap,img) & img.integralImage(tilted)\\[0.3cm]
  Convolution performs a shape 
  change on an image. & & Filter2D(Bitmap,retVal,myKernel,center) & img.convolve(,kernel,center)\\[0.3cm]
  Function searches an image \\ 
  for a template image &  &  & img.findTemplate(template\_image, threshold, method)\\[0.3cm]
 \end{tabular} 
\begin{tabular}{llll}
  \hline
  Description & Matlab & OpenCV & SimpleCV \\ \hline \\[.1cm] 
  Return any text it can find \\
  using OCR on the image & & &img.readText()\\[0.3cm]
  extract perfect circles\\ 
  from the image & & & img.findCircle(canny,thresh,distance)\\[0.3cm] 
  Attempts to perform automatic\\ 
  white balancing& & &img.whiteBalance(method)\\[0.3cm]
  Apply a LUT (look up table)\\
  to the pixels in a image & &LUT(bitmap,bitmap,fromarray(LUT)) &img.applyLUT(rLUT,bLUT,gLUT)\\[0.3cm]
  Finds keypoints in an image and &&&img.\_getRawKeypoints(thresh,flavor,\\
  returns them as the raw keypoints  & & & highQuality, forceReset)\\[0.3cm]
  Method does a fast local approximate\\ 
  nearest neighbors (FLANN) calculation& & & img.\_getFLANNMatches(sd,td)\\ 
  between two sets of feature vectors\\[0.3cm]
  Calculates keypoints for both images,& & & img.drawKeypointMatches(template,\\ 
  determines the keypoint correspondences& & & thresh, minDist,width)\\[0.3cm]
  Match a template image with another & & &img.findKeypointMatch(template,\\ 
  image using SURF keypoints.& & &quality,minDist,minMatch)\\[0.3cm]
  This method finds keypoints in an image & & & img.findKeypoints(min\_quality,\\ 
  and returns them as a feature set& & & flavor,highQuality)\\[0.3cm]
  Performs an optical flow calculation\\
  and attempts to find motion& & CalcOpticalFlowHS(previousFrameGrayscaleBitmap, & img.findMotion(previous\_frame,\\
  between two subsequent frames & & imgGrayscaleBitmap,block,shift,spread,0,xf,yf)& window, method, aggregate) \\[0.3cm]
  Returns the colors in the \\
  palette of the image & & & img.getPalette(bins,hue)\\[0.3cm]
  \end{tabular} 

  \begin{tabular}{llll}
  \hline
  Description & Matlab & OpenCV & SimpleCV \\ \hline \\[.1cm] 
  Takes in the palette from  \\ 
  another image and attempts to &&&img.rePalette(palette,hue)\\ 
  apply it to this image\\[0.3cm]
  returns the visual representation \\
  (swatches) of the palette &&&img.drawPaletteColors(size,horizontal,bins,hue)\\
  in an image \\[0.3cm]
  The method then goes through \\
  and replaces each pixel with \\
  the centroid of the clutsters &&&img.palettize(bins,hue)\\ 
  found by k-means \\[0.3cm]
  Palettization and behaves &&&img.findBlobsFromPalette(palette\_selection,\\ 
  similar to the fndBlobs &&& minsize, maxsize)\\[0.3cm]
  Method uses the color palette\\ 
  to generate a binary image& & & img.binarizeFromPalette(palette\_selection)\\[0.3cm]
  Skeletonization of the image&&&img.skeletonize(radius)\\[0.3cm]
  smartThreshold uses a method \\
  graph cut, to automagically &&grabCut(npimg,mask,rect,tmp1,tmp2,10,mode)&img.smartThreshold(mask, rect)\\
  generate a grayscale mask image\\[0.3cm]
  It takes a image converts it \\
  to grayscale, and applies a threshold&&&img.smartFindBlobs(mask,rect,thresh\_level)\\[0.3cm]
  This method is same as Paint\\
  bucket tool in image&& FloodFill(bmp,tuple(points),color,&img.floodFill(points,tolerance,color,\\
  manipulation program&&lower,upper,flags)&lower,upper,fixed\_range)\\[0.3cm]
  Returns Image where the values \\
  similar to the seed pixel have & & &img.floodFillToMask(points,tolerance,\\
  been replaced by the input color& & & color,lower,upper,fixed\_range,mask)\\[0.3cm]
  
  \end{tabular} 
 
  \begin{tabular}{llll}
  \hline
  Description & Matlab & OpenCV & SimpleCV \\ \hline \\[.1cm] 
 A featureset of blobs & & & img.findBlobsFromMask(mask,threshold=128,\\
 form the Mask Image & & & minsize=10, maxsize=0 )\\[0.3cm]
  Returns the RAW DFT \\
  transform of an image&&DFT(src, dst,CV\_DXT\_FORWARD)&img.rawDFTImage(grayscale)\\[0.3cm]
  Returns the log value of \\
  the magnitude image of &&&img.getDFTLogMagnitude(grayscale)\\ 
  the DFT transform \\[0.3cm]
  Apply an arbitrary filter \\
  to the DFT of an image& & & img.applyDFTFilter(flt,grayscale)\\[0.3cm]
  Applies a high pass DFT filter & & & img.highPassFilter(xCutoff,yCutoff,grayscale)\\[0.3cm]
  Applies a low pass DFT filter&&&img.lowPassFilter(xCutoff,yCutoff,grayscale)\\[0.3cm]
  Method applies a simple &&& img.bandPassFilter(xCutoffLow, xCutoffHigh,\\
  band pass DFT filter&&& yCutoffLow, yCutoffHigh,grayscale)\\[0.3cm]
  Method performs an inverse \\
  discrete Fourier transform &&& InverseDFT(raw\_dft\_image)\\[0.3cm]
  Creates a butterworth filter\\ 
  of 64x64 pixels, resizes it &&&img.applyButterworthFilter(dia,order,highpass,grayscale)\\
  to fit the image \\[0.3cm]
  Creates a gaussian filter \\
  of 64x64 pixels, resizes  &&&img.applyGaussianFilter(dia, highpass, grayscale)\\
  it to fit image \\[0.3cm]
  DFT is applied on image \\
  using gaussian lowpass filter&&&img.applyUnsharpMask(boost,dia,grayscale)\\[0.3cm]
  \end{tabular} 
 
 \end{flushleft}
\end{document}
